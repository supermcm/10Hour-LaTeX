\begin{abstract}
The summary is an essential part of your MCM/ICM paper. The judges place considerable weight on the summary, and winning papers are often distinguished from other papers based on the quality of the summary.

To write a good summary, imagine that a reader will choose whether to read the body of the paper based on your summary: Your concise presentation in the summary should inspire a reader to learn about the details of your work. Thus, a summary should clearly describe your approach to the problem and, most prominently, your most important conclusions.  Summaries that are mere restatements of the contest problem, or are a cut-and-paste boilerplate from the Introduction are generally considered to be weak.

Besides the summary sheet as described each paper should contain the following sections:
\begin{itemize}
\item \textbf{Restatement and clarification of the problem}: State in your own words what you are going to do.
\item \textbf{Explain assumptions and rationale/justification}: Emphasize the assumptions that bear on the problem. Clearly list all variables used in your model.
\item \textbf{Include your model design and justification} for type model used or developed.
\item \textbf{Describe model testing and sensitivity analysis}, including error analysis, etc.
\item \textbf{Discuss the strengths and weaknesses} of your model or approach.
\end{itemize}

\end{abstract}
